\documentclass[12pt,a4paper,onecolumn]{article}

\usepackage[a4paper,left=1.1cm, right=1.0cm, top=2.0cm,bottom=1.7cm]{geometry}

%----------------------------------------------------------------------
%Auxiliary libraries.
%----------------------------------------------------------------------

\usepackage{listings}
\usepackage{amssymb}
\usepackage{amsmath}
\usepackage{indentfirst} %To use indentation of first paragraph.
\usepackage{soul} %To use command \ul for introducing underline text.
\usepackage{xcolor}
\usepackage{verbatim} %To use \begin{comment}
%\usepackage[utf8]{inputenc}
%\usepackage{multirow} %To combine cells of table vertically.
%\usepackage{fancyhdr} %To use header.
%\usepackage{seqsplit} %To use command \seqsplit for splitting lines.
%\usepackage{enumitem}
%\usepackage{hyperref} %To activate cross references.
%\usepackage{graphicx} %To include figures.
%\usepackage{subfigure} %To include subplots.

%----------------------------------------------------------------------
%Adjust parameters of listings (colorful text, background, etc.).
%----------------------------------------------------------------------

\lstset{basicstyle=\ttfamily,
  showstringspaces=false,
  %basicstyle=\small\ttfamily, %To make font smaller.
  %breaklines=false,
  columns=fullflexible, %To make text without whitespaces round a single dot.
  %Also possible columns=flexible or columns=fixed,
  backgroundcolor=\color{gray!25},
  commentstyle=\color{red},
  keywordstyle=\color{blue}
}

%======================================================================
%Start of document.
%======================================================================

\begin{document}

%----------------------------------------------------------------------
%Start document and set title, content, etc.
%----------------------------------------------------------------------

\title{\huge Manual for generating Plots of Monitoring on the Lichtenberg Cluster}
\author{Mikhail Ovsyannikov}
\date{24.09.2018}
\maketitle

\newpage

\tableofcontents

\newpage

%======================================================================
%Section: Access to Monitoring Node and Repository.
%======================================================================

\section{Access to Monitoring Node and Repository}
\label{sec:access_monitor_tools}

%======================================================================

Data for generating plots (log-files) are generated by using scripts executed on a \textbf{monitoring node} with the name \lstinline{hxi0001}.
The monitoring node is in turn accessed from a so-called \textbf{master node} to which the user has to log in:
\begin{lstlisting}
ssh -C -A -i <pathToPrivateKey> <username>@lmaster3.hrz.tu-darmstadt.de
\end{lstlisting}
From the master node, the monitoring node is accessible by the command
\begin{lstlisting}
ssh hxi0001
\end{lstlisting}

\vspace{\baselineskip}

\underline{\textbf{Attention}}: The pair of the public and private keys is needed, and the public key must be stored on the cluster.
This step \ul{must be discussed with the administrators of the Lichtenberg cluster} as well as the creation of the account on the master/monitoring node.
\vspace{\baselineskip}

All tools and scripts applied for generating the log-files and the plots are available in a git-repository.
\vspace{\baselineskip}

\underline{\textbf{Attention}}: The pair of the public and private keys is needed, and the public key must be stored on the server of the repository.
This step \ul{must be discussed with the members of the HKHLR center}.
\vspace{\baselineskip}

When the public key is located correctly, the repository can be cloned by the following command:
\lstinputlisting[language=bash, label={list:git_clone}]
{./Sources/command_clone.txt}
It is enough to clone one time and to update further.
Some programs and scripts of the monitoring repository are executed on a \textbf{login-node} of the cluster.
Thus, the repository can be updated on the local machine and copied into a directory of the login-node by the \lstinline{scp} command.
Another way is to clone/update the repository directly within the login-node, but in the manual the first way is more preferable, when the repository is copied from the local machine to the login-node after updating.

In addition, some scripts must be executed on the \textbf{monitoring node} and must be copied into a directory of the \textbf{master node}.
Note, \ul{everything located on the master node is available from the monitoring node}, but the latter is accessible from outside.

%======================================================================
%Section: Monitoring tools.
%======================================================================

\section{Monitoring Tools}
\label{sec:monitoring_tools}

%----------------------------------------------------------------------

\begin{enumerate}
%\setlength{\itemsep}{\baselineskip}

\item Update the repository of the tools for monitoring on the local machine or on the cluster:
\begin{lstlisting}
git pull
\end{lstlisting}
The repository must be cloned before.

\item Go to the directory of the updated repository which contains the main tools:
\begin{lstlisting}
<pathToRepo>/monitoring/Evaluation/MonitoringPlots/helper/Darmstadt
\end{lstlisting}

The directory must contain the important bash-scripts applied for generating log-files and reports:
\begin{itemize}
  \item \lstinline{get_SLURM_data.sh} for generating log-files for weekly plots;
  \item \lstinline{get_data.sh} for generating log-files for monthly plots.
\end{itemize}

\item Copy the aforementioned bash-scripts into the master node described in section \ref{sec:access_monitor_tools}.

Everything located in the master nodes is visible and accessible from the monitoring node.

\item Copy the updated repository into a directory accessible from a login-node.

If the repository is updated directly on the cluster, copying is not needed.

Further work is done by using the account of the \underline{login-node}.

\item Log in to the login-node and load the required modules:
\lstinputlisting[language=bash, label={list:module_list}]
{./Sources/module_monitor.txt}
\vspace{\baselineskip}

\underline{\textbf{Remark}}: The module for git is needed, if the repository is updated/cloned from the login-node.
\vspace{\baselineskip}

\item Go to the directory of the copied repository which contains the main tools:
\begin{lstlisting}
<pathToRepo>/monitoring/Evaluation/MonitoringPlots/helper/Darmstadt
\end{lstlisting}

\item Compile the following programs by using the make command in the aforementioned directory:
\begin{itemize}
  \item \lstinline{HKHLR_DataMiner.exe} for generating reports for weekly plots;
  \item \lstinline{process_slurm_logs.exe} for generating  monthly plots.
\end{itemize}

\item Create a new directory used for generating plots on the login-node, e.g. \lstinline{Plots} used further.

\item Copy the executable \lstinline{HKHLR_DataMiner.exe} into the directory for plots, i.e. \lstinline{Plots}.

\item Copy the scripts for generating plots from the directory
\begin{lstlisting}
<pathToRepo>/monitoring/Evaluation/MonitoringPlots/scripts
\end{lstlisting}
into the directory all plots are to be generated, i.e. \lstinline{Plots}.

The following scripts are needed:
\begin{lstlisting}
ClusterAnalysis.MonthlyEfficiencyStatistics.plot
ClusterAnalysis.NumberOfJobsPerMonth.plot
ClusterAnalysis.NOJPM_Comparison.plot
ClusterAnalysis.ProjectsStatistics_MK2.plot
ClusterAnalysis.ProjectsStatistics.plot
generate_project_reports.sh
generate_plots.sh
\end{lstlisting}

\end{enumerate}

%======================================================================
%Section: Weekly Plots of Power Projects of Top-list.
%======================================================================

\section{Weekly Plots of Power Projects of Top-list}
\label{sec:weekly_plots_top_power_projects}

%----------------------------------------------------------------------
%Subsection: Generation of weekly plots.
%----------------------------------------------------------------------

\subsection{Generation of weekly plots}
\label{sec:weekly_plots_generation}

%----------------------------------------------------------------------

\begin{enumerate}

\item Log in to the \lstinline{hxi0001} node (see section \ref{sec:access_monitor_tools}) for generating log-files for \textbf{weekly plots}.

\item Run the \lstinline{get_SLURM_data.sh} script to generate the log-files by the command of the general format 
\begin{lstlisting}
./get_SLURM_data.sh YYYY-MM-DD_1 YYYY-MM-DD_2
\end{lstlisting}
where \lstinline{YYYY-MM-DD_1} and \lstinline{YYYY-MM-DD_2} are the dates which specify the period of time for collecting data.
Each generated file corresponds to monitoring data of one week.

For example, the log-file of week 33 is generated by using the command
\begin{lstlisting}
./get_SLURM_data.sh 2018-08-13 2018-08-19
\end{lstlisting}
The generated file have the name \lstinline{HKHLR_SLURM_Darmstadt_2018-W33.log} which contains the corresponding year (2018) and the week (33).

The common form of the log file can be written as follows:
\begin{lstlisting}
HKHLR_SLURM_Darmstadt_YYYY-WN1.log
\end{lstlisting}
where \lstinline{WN1} specifies the week which consists of the letter \lstinline{W} and the corresponding week number \lstinline{N1}, \lstinline{YYYY} specifies the corresponding year.

\item Copy the log-files generated for particular weeks into the directory for generating plots, i.e. \lstinline{Plots}.

\begin{enumerate}
\item Copy data \textit{from} the master node \textit{into} the local machine.
\item Copy data \textit{from} the local machine \textit{into} the login-node.
\end{enumerate}

Further work is done on the \underline{login-node} of the cluster.

\item Log in to the login-node and go to the directory for plots, i.e. \lstinline{Plots}.

\item Run \lstinline{HKHLR_DataMiner.exe} to generate a so-called \textbf{report-file}:
\begin{lstlisting}
./HKHLR_DataMiner.exe WN1 WN1 HKHLR_SLURM_Darmstadt_YYYY-WN1.log
\end{lstlisting}

The command can be applied for processing several log-files which correspond to several weeks:
\begin{lstlisting}
./HKHLR_DataMiner.exe WN1 WN1 HKHLR_SLURM_Darmstadt_YYYY-WN1.log \
                      WN2 WN2 HKHLR_SLURM_Darmstadt_YYYY-WN2.log
\end{lstlisting}

The example of the command applied for the log-file generated for week 33 of 2018:
\begin{lstlisting}
./HKHLR_DataMiner.exe W33 W33 HKHLR_SLURM_Darmstadt_2018-W33.log
\end{lstlisting}

After executing, the directory for generating plots contains a lot of files on performance and efficiency of every project.
The summary on the efficiency of all the project is given in the file
\begin{lstlisting}
HKHLR_SLURM_Darmstadt_YYYY-WN1.log.report
\end{lstlisting}

For the example of the log-file generated for week 33 of 2018, the report-file is called
\begin{lstlisting}
HKHLR_SLURM_Darmstadt_2018-W33.log.report
\end{lstlisting}
The names of the projects are given in the decreasing order according to the overall efficiency estimated for every project.
Thus, projects of the list of the top of $N$ Power Projects or simply a top-list (e.g. top 5 or top 10) can be easily determined by the first $N$ projects listed in the report-file.

\item Generate $N$ sets of plots of efficiency for the first $N$ projects of the list in the file-report.

Each set of plots is generated by the script \lstinline{generate_project_reports.sh} as follows:
\begin{lstlisting}
./generate_project_reports.sh Lichtenberg name_of_history_file
\end{lstlisting}
\vspace{\baselineskip}

\underline{\textbf{Attention}}: The first argument of the above command, \lstinline{Lichtenberg}, \underline{must be present}, and it is added to titles of plots.
\vspace{\baselineskip}

All the $N$ sets can be generated by the one command.
For example, the five most ``powerful'' projects written listed in the report-file:
\begin{lstlisting}
Project project00696 12.2
Project project00770 9
Project project00689 5.5
Project project00574 4.9
Project project00660 4.7
\end{lstlisting}
The corresponding command for generating figures of the plots:
\begin{lstlisting}
./generate_project_reports.sh Lichtenberg Project.project00696.W33.hist.dat \
			                  Project.project00770.W33.hist.dat \
					  Project.project00689.W33.hist.dat \
					  Project.project00574.W33.hist.dat \
					  Project.project00660.W33.hist.dat
\end{lstlisting}

After executing the above command, the figures are located in the new directory called \lstinline{plots}.

\item Download the obtained figures to the local machine and send them for a further analysis.

All the generated files excluding the \textbf{plots}, the \textbf{log-file} and the \textbf{report-file} \underline{can be deleted}.

The \underline{next step} is to create tickets for all the projects of the top-list to be analysed in RedMine.

\end{enumerate}

%----------------------------------------------------------------------
%Subsection: Tickets for projects of the top-list.
%----------------------------------------------------------------------

\subsection{Tickets for projects of the top-list}
\label{sec:tickets_powerusers}

%----------------------------------------------------------------------

Tickets are created in the \textbf{RedMine} system for all the projects of the top-list generated weekly and described in section \ref{sec:weekly_plots_generation}.
Those projects which have some problems with performance and efficiency observed for a couple of weeks have to be described carefully (a name of a professor, a field of investigation, code, an so on).
One can obtain a detailed description of such a project in an office of administrators of the Lichtenberg cluster.
A special attention should be paid to projects which belong to the LARGE class.
Projects without explicit problems regarding performance and efficiency can be only mentioned and described very briefly (it is enough to write that ``computational resources of the cluster are used properly'' or so).
The obtained documents must be returned back.
\vspace{\baselineskip}

\underline{\textbf{Attention}}: Personal information of the user \underline{must not be written in a ticket}.
Only the name of a professor or a leader, the field of investigation, a description of code, etc. can be added to the ticket.
\vspace{\baselineskip}

\underline{The following steps are done in RedMine before creating new tickets}.

\begin{enumerate}

\item Select \textit{``Beratungsprotokoll''} in the list \textit{``Zu einem Projekt springen''} in the right top corner.

\item Select the tab \textit{``Tickets''}.

\item Apply sequentially the two filters chosen in the list \textit{``Filter hinzuf\"ugen''}:
\begin{enumerate}
\item filter \textit{``Tracker''};
\item filter \textit{``Standort''};
\end{enumerate}

\item Select in one list \textit{``PowerProject''} and in another one \textit{``Darmstadt''} (both are located near the center).

\item Press \textit{``Anwenden''}.

After completing the above actions, old tickets are show in the list, and new ones can be created.

\item Press \textit{``Neues Ticket''} to start creating a new ticket.

\end{enumerate}

\underline{Each ticket is created according to the same scheme after pressing \textit{``Neues Ticket''}}.

\begin{enumerate} 

\item Select \textit{``PowerProject''} as the option for \textit{``Tracker''}.

\item Enter the title of a new ticket.
\vspace{\baselineskip}

\underline{\textbf{Attention}}: Titles of all tickets of projects of the top-list \underline{must have the same format (or style)}.
Only the project ID and the week number are changed.
\vspace{\baselineskip}

\item Describe the project, if the plots of its efficiency show some problems with performance.

It can be observed during several weeks.

\item Select \textit{``Neu''} as the option of \textit{``Status''}, if the ticket is new or does not have any problems with efficiency.

\item Select the priority of the ticket in \textit{``Priorit\"at''} (\textit{``Normal''} by default).

This parameter is important, if there are some problems with efficiency,
and depends on the size of a project and problems on performance detected.

\item Select \textit{``Standort''} and \textit{``Standort (Nutzer)''}, if they are known (\textit{``Darmstadt''} by default).

These parameters are important, if there are some problems with efficiency.

\item Enter the department in the field \textit{``Fachbereich''} (`\textit{``No Information''} by default).

This field is important, if there are some problems with efficiency.

\item Enter the name of a professor in the field \textit{``Zugeh\"origer Prof''} (`\textit{``No Information''} by default).

This field is important, if there are some problems with efficiency.

\item Select the cluster \textit{``Lichtenberg''} as the option of \textit{``Rechner''}.

\item Select the rank in the field \textit{``Rank''} according to the position of the project in the top-list.

\item Enter the score in the field \textit{``Prozentuale Nutzung''} which is found in the file of the top-list.

\item Other fields are filled in according to the provided information, if there are some problems with efficiency.

\item Press the button \textit{``Browse''} after filling in all the fields

The set of the plots on performance and efficiency must be uploaded.

\item Press the button \textit{``Anlegen und weiter''} to create the current ticket and to start a new one.

\item If there is a ticket created earlier for the same project, the reference to the new ticket (its number or ID) should be added to the first earliest ticket.

\begin{enumerate}
\item Find and open the earliest ticket for the same project.
\item Press \textit{``Hinzuf\"ugen''} in the field called \textit{``Zugeh\"orige Tickets''},
\item Enter the number of the ticket created recently.
\end{enumerate}
\vspace{\baselineskip}

\underline{\textbf{Remark}}: Each week the tickets must be updated in Redmine.
When the user does not need any help and uses computational resources properly, it is not needed to add any detailed information except only the plots and the \% of usage.
\vspace{\baselineskip}

\end{enumerate} 

%======================================================================
%Section: Monthly Plots of Monitoring.
%======================================================================

\section{Monthly Plots of Monitoring}
\label{sec:monthly_plots}

%======================================================================

\begin{enumerate}

\item Log in to the \lstinline{hxi0001} node (see section \ref{sec:access_monitor_tools}) for generating log-files for \textbf{monthly plots}.

\item Prepare the script \lstinline{get_data.sh} for a particular month.

The file contains lines with the \lstinline{sacct} command executed for each month which is limited by the days of the start and of the end of the month.
These days are specified by the options \lstinline{-S} and \lstinline{-E} and are written in the format \lstinline{YYYY-MM-DD}.
As an example the following part of the line is used for data for January 2017:
\begin{lstlisting}
sudo sacct -n -P -S2017-01-01 -E2017-01-31... > 2017_01.log
\end{lstlisting}
It is seen that the output of the command above is directed into the log-file with the name \lstinline{2017_01.log}, which should be in agreement with the month specified by the options \lstinline{-S} and \lstinline{-E}.
Thus, \underline{the parameters of the line to be changed}:
\begin{itemize}
\item the dates specified by the options \lstinline{-S} and \lstinline{-E} (especially \textbf{years});
\item the name of the file of the output (for consistency).
\end{itemize}

\underline{The next change of the script-file is connected with its structure}: if the monthly plots are to be generated for a particular month, the file must include the line with the \lstinline{sacct} command for this month and the lines of the command for all the previous months of the current year (starting from January).

For example, the script for generating log-files for February 2017 must contain the following two lines:
\begin{lstlisting}
sudo sacct -n -P -S2017-01-01 -E2017-01-31... > 2017_01.log
sudo sacct -n -P -S2017-02-01 -E2017-02-31... > 2017_02.log
\end{lstlisting}

\underline{\textbf{Attention}}: The other options of the \lstinline{sacct} command must not be changed.

\underline{\textbf{Remark}}: The generation of log-files can take significant time.

\item Run the \lstinline{get_data.sh} to generate the log-files (without any arguments):
\begin{lstlisting}
./get_data.sh
\end{lstlisting}

After executing, there are log-files generated for every month according to the number of the lines with the \lstinline{sacct} command written in the script-file (the results of the output).

\item Copy the obtained log-files from the \lstinline{hxi0001} node into the directory for plots accessible from a login node (usually in two steps with using the local machine).

Further work is done on the \underline{login node} of the cluster.

\item Log in to the login node and go to the directory for generating the plots, i.e. \lstinline{Plots}.

\item Extract data for plots from the uploaded log-files by the executable \lstinline{process_slurm_logs.exe}.

There must be \underline{three command-line arguments} passed in the program:
\begin{enumerate}
\item the in-file label of the month written as \lstinline{"YYYY/MM"};
\item the file-system label of the month written as \lstinline{"YYYY_MM"};
\item the name of the corresponding log-file, which is usually \lstinline{YYYY_MM.log}.
\end{enumerate}
\vspace{\baselineskip}

\underline{\textbf{Attention}}: The quotation marks of the arguments in 1 and 2 are necessary.
\vspace{\baselineskip}

The general form of the command and its arguments looks as follows
\begin{lstlisting}
./process_slurm_logs.exe "YYYY/MM" "YYYY_MM" YYYY_MM.log
\end{lstlisting}

Moreover, for generating plots for a particular months, \ul{data of the previous months of a current year must be included}.
This can be done by passing the three arguments for every month when the program \lstinline{process_slurm_logs.exe} executed one time:
\begin{lstlisting}
./process_slurm_logs.exe "YYYY_1/MM_1" "YYYY_1_MM_1" YYYY_1_MM_1.log \
                        "YYYY_2/MM_2" "YYYY_2_MM_2" YYYY_2_MM_2.log
\end{lstlisting}
The second way is to execute the program \lstinline{process_slurm_logs.exe} multiple times:
\begin{lstlisting}
./process_slurm_logs.exe "YYYY_1/MM_1" "YYYY_1_MM_1" YYYY_1_MM_1.log
./process_slurm_logs.exe "YYYY_2/MM_2" "YYYY_2_MM_2" YYYY_2_MM_2.log
\end{lstlisting}
\vspace{\baselineskip}

\underline{\textbf{Attention}}: If the program is executed multiple times, the generated data can be removed only \ul{after the final execution of the program}.
\vspace{\baselineskip}

The example of the command for generating plot for August 2018 (taking into account the data starting from January 2018):
\begin{lstlisting}
./process_slurm_logs.exe "2018/01" "2018_01" 2018_01.log \
                         "2018/02" "2018_02" 2018_02.log \
			 "2018/03" "2018_03" 2018_03.log \
			 "2018/04" "2018_04" 2018_04.log \
			 "2018/05" "2018_05" 2018_05.log \
			 "2018/06" "2018_06" 2018_06.log \
			 "2018/07" "2018_07" 2018_07.log \
			 "2018/08" "2018_08" 2018_08.log
\end{lstlisting}
\vspace{\baselineskip}

\underline{\textbf{Remark}}: It is better to write the aforementioned command in one line due to multiple arguments being passed (to make it briefer it is broken into the several lines using ``$\backslash$'' in the above listing).
\vspace{\baselineskip}

\item Draw figures of the plots based on the extracted data by using the script \lstinline{generate_plots.sh}:
\begin{lstlisting}
./generate_plots.sh Darmstadt
\end{lstlisting}
The argument \lstinline{Darmstadt} specified in the command is added to the titles of the generated plots, and if it is not given, it is replaced by the word ``unknown'' or so (by default).

After executing the script, the directory \lstinline{plots} is created, and all the generated figures are located there (if it was created before, the figures are added to the old content).

\item Add the obtained plots to the monitoring slides in the \textbf{ShareLatex project}.

The most important figures have the following general names:
\begin{itemize}
\item the file ``YYYY\_MM.perJob.EfficiencyStatistics.png'';
%\item the file ``YYYY\_MM.perJob.EfficiencyStatistics.internal.png'';
\item the file ``YYYY\_MM.perJob.UnusedCPUHours.png'';
\end{itemize}
The prefix ``YYYY\_MM'' consists of a corresponding year and a month.
These two files are generated for every month and correspond to the monthly log-files (usually starting from January of the current year).
Thus, they can be repeated, and only the figures generated for the new months should be uploaded in Sharelatex.

In addition, the important figures with the summary information of all the processed months have the following fixed names:
\begin{itemize}
\item the file ``JobSummaryStatistics.10CPUMi.png'';
\item the file ``JobSummaryStatistics.1CPUHr.png'';
\item the file ``JobSummaryStatistics.All.png''.
\end{itemize}
\ul{These three files are new and must be updated every time in ShareLatex}.

The structure of the slides of the project is typical and can be copied for adding new slides.
Some general information or a description should be also written on the new slides.

\end{enumerate}

%======================================================================
%End of document.
%======================================================================

\end{document}

%======================================================================
